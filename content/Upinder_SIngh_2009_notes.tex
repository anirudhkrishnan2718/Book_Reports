\chapter{History of ancient and early medieval India}

\section*{Introduction : Ideas of the Early Indian Past}
\begin{itemize}
    \item Book Citation : \cite{Upinder_Singh_2009}
    \item Subcontinental India (referring to modern day India, Pakistan, Sri Lanka, Nepal, Bhutan, and Bangladesh) can be divided into major geographical regions based on the dominant river systems, mountain ranges or plateaus.
    \item Movement of goods and people between India's geographical regions and outside the subcontinent has been observed since ancient times.
    \item \Gls{prehistory} ended with the invention of the Mesopotamian Cuneiform script (3400 BCE) and Egyptian Hieroglyphics (3100 BCE), and the Harappan script (3000 BCE)
    \item Since the Harappan script has not yet been deciphered, \gls{protohistory} in the context of India refers to the period 1500 BCE - 500 CE, which has an oral literature but no direct written sources.
    \item \Gls{histriography} in India began with the Orientalist stage of early 18th and 19th century British India, with the founding of the \gls{asi} in 1871 characterized by a western-centric critique of Indian customs and traditions, an over-reliance on Brahmanical interpretations of scripture.
    \item Spurred on by the national freedom movement, nationalist historians contributed to Indian histriography by glorifying India's past as a Golden Age to be ended by the arrival of Muslim invaders.
    \item Marxist historians made the perspective shift of dividing history into periods based on class-relations and agrarian norms, instead of the prevailing political stratification.
    \item Historical narratives have suffered from textual sources being the primary driver with archaeological sources only being used as corroborating evidence.
    \item Most of the people in India's ancient past are not represented in its history because of the monopoly of the literate and elite few of the time in recording written information. Archaeological data becomes the main way to tell the stories of these forgotten populations.
\end{itemize}

\section{Understanding Literary and Archaeological Sources}

\subsection{Reading Ancient Texts from a Historical Point of View}
\begin{itemize}
    \item Unlike China, which evolved paper by the 3ed century BCE and wood-block printing by 4th century BCE, India still had a vibrant tradition of palm leaf manuscripts all the way from antiquity to the 19th century when the printing press finally superseded it.
    \item Texts have been composed over centuries of oral tradition before they were compiled, finalized and written down, which makes historical context of the physical manuscript much narrower than the work being written down.
    \item The Vedas (written in Sanskrit), which are the oldest surviving Indian text belong to the Indo-Iranian branch of the Indo-European family of languages. The other major linguistic family in India is the Dravidian family, which covers all south Indian modern day languages.
    \item In Ancient India, there was no delineation between the religious and secular teachings contained within a religion (such as \Gls{dharma} or the Buddhist \Gls{dhamma})
    \item The four Vedas (dated at about 1500 BCE - 500 BCE) are the world's oldest existing poetry which are primarily a guide to religious rituals and the philosophical interpretations of such rituals. They are religious texts, with very few references to historical events, and thus, cannot be used to infer much about the sociology or history of ancient Northern and North-western India.
    \item Mahabharata (composed over 400 BCE - 400 CE) and the Ramayana (composed over 500 BCE to 300 CE) are the two epics of ancient India. Characters, location and events from the two epics are mentioned in the other, showing that these were not isolated works.
    \item Events and characters described in the two epics have not been verified to be true. They may simply have been small scale conflicts glorified by bards and poets over the centuries.
    \item Puranas, Dharmashastras, and other Vedanga texts solidify Hindu traditions such as the stratification of society, the rights of women, and other prescriptive actions to be undertaken in the quest for \Gls{moksha}. Contradictions and divergent commentaries within the Vedangas are a result of the many centuries over which these works were composed.
    \item Buddhist, Jain, early Tamil (Sangam), Kaanada and Tulu literature provides many religious, philosophical, scientific, technical and daramatic works to draw historical information from. These works, however, need to be interpreted carefully because of the nature of the patrn-artist relationship under which most of them were produced.
    \item Accounts of foreign travelers from China (Fa Hien and Hiuen Tsang), Persia (Al-Biruni) and Greece (Megasthenes) are a valuable window into the life and times of ancient and early medieval India.
\end{itemize}

\subsection{Archaeology and the Early Indian Past}
\begin{itemize}
    \item Material evidence found at a site often depends on the forces of nature that may or may not have destroyed part of it (usually the organic artefacts), the movement of human settlements the centuries as a result of natural disasters, and their \gls{stratigraphy} within the archaeological site.
    \item In India, underwater archaeology deals with entire cities that have been submerged over the centuries (such sa Dwaraka, Gujarat) requiring modern advances in remote sensing, radar, and diving.
    \item Over the last 50 yars, the advent of \gls{archaeometry} (most famously Carbon-14 dating) has enabled a much less intrusive and hands-off method of analysing artefacts, especially those buried deep underground or undersea.
    \item \Gls{paleontology} and the related field of paleo-pathology, which looks at the diseases and deaths of past human societies are also enabled by advances in modern medicine.
    \item In India, more-so than the rest of the world, ancient crafts, techniques and workmanship survives, making the study of these modern day craftsmen a useful tool in piecing together the picture of unearthed ancient artefacts.
\end{itemize}

\subsection{Epigraphy: The Study of Inscriptions}
\begin{itemize}
    \item \Gls{epigraphy} in India starts with the un-deciphered Harappan script, with the 400 BCE Brahmi script around the time of Ashoka being the earliest fully deciphered one.
    \item In a script, a written symbol stands for a word (logographic), a syllable (syllabic) or a sound (alphabetic). The last category requires that vowels exist fully independent of consonants.
    \item The Brahmi script developed from 300 BCE to 600 CE evolved into the Devanagari script (around 1000 CE) and proto-Bengali in eastern India (around 1200 CE).
    \item Modern Tamil script took shape around 700 CE emerging from local varieties of Brahmi, around the time of the Pallava dynasty. Other South Indian scripts evolved by 1400 CE.
    \item Scholars working for the East India Company in the 1800s deciphered the Brahmi and Kharoshti scripts painstakingly over many years, with the assistance of bi-script documents acting as Rosetta stones.
    \item Sanskrit subsumed Prakrit as the language of high culture and learning by 500 CE, along with Tamil in South India achieving the same status by 700 CE
    \item Dates on inscriptions usually were eras based on the ascension of Kings, or the start of dynasties, along with the luni-solar Hindu calendar.
    \item Most inscriptions tend to be dedication sot great men and women, memorials of martyrs and soldiers, records of land or wealth donated by kings, or simple records of the existing political, social and economic climate made by common folk.
\end{itemize}

\subsection{Numismatics: The Study of Coins}
\begin{itemize}
    \item Since coins from the same time degrade in size and weight the longer they circulate, \gls{metrology} offers an immediate means of dating the sites at which they are found.
    \item The earliest \gls{coinage} in India appeared around 400 BCE, usually made of silver or copper even though barter still continued as an alternative medium of commerce.
    \item Coins of low value often used copper, lead, tin, bronze or other alloys with silver and were issued either by royal decree or by private guilds.
    \item \Glspl{legend} on coins are rarely dates or other chronological information. They are usually observed to be the names of kings, the mint towns where they were produced, or a religious inscription.
    \item \Gls{counter-struck} coins are valuable sources of information about political rivalries, conquest and succession, and the geographical spans of kingdoms in ancient and early medieval India
\end{itemize}

\section{Hunter-Gatherers of the Palaeolithic and Mesolithic Ages}

\subsection{The Geological Ages and Hominid Evolution}
\begin{itemize}
    \item Darwin's theories in the 1850s contribued to a paradigm shift in our understanding of the emergence of the Human species and broke the stranglehold of \Gls{creationism}.
    \item The fact that life on Earth appeared long before humans, that some species of primate were close genetic relatives of humans, and that humans are not some end-goal of the process of evolution were the primary breakthroughs of Darwin and Thomas Huxley.
    \item Biological evidence for the evolution of humans follows the trail of increasing brain size, changes in the pelvic bone, changing food habits, and the advent of walking upright on two legs.
    \item \textit{Homo habilis} is the earliest known species of the Human genus to have used tools, with sites in Keyna and Tanzania indicating tool manipulation as early as 2.5 \gls{mya}.
    \item \textit{Homo sapiens} eventually replaced all other species and migrated out of the African continent around 0.2 mya, which is dated as the time of anatomically modern humans appearing.
    \item Culturally modern humans who had a sense of individual and group identity, practiced art and rituals, developed extensive communication, and advanced tool making only seem to have appeared 50,000 years ago.
\end{itemize}

\subsection{Hominid Remains in the Indian Subcontinent}
\begin{itemize}
    \item The earliest hominid fossils found in India  - dated to 0.3 mya - correspond to the \Gls{paleolithic age} and represent either early \textit{Homo sapiens} or the end stages of the other Human species.
    \item The antiquity of most hominid remains found in India under the aegis of the ASI between 1982 - 2007 is uncertain, especially the remains found in the Narmada basin in Madhya Pradesh.
    \item Primate fossils dated to 13 mya have been found in the Shivalik hill range in the Himalays over the last century, with similar remains also observed in other continents.
\end{itemize}

\subsection{Palaeo-environments}
\begin{itemize}
    \item Between 50 to 35 mya, the Indian landmass, which had broken off from unified Gondwanaland, finally collided with the Asian landmass resulting in the uplift of the Himalayas and Tibetan plateau.
    \item \Gls{plate tectonics} is responsible for the fertile alluvial plains of North India and its perennial river system, thus laying the foundation for a flourishing human civilization.
    \item Around 10,000 years ago, the Holocene Era, which established all of the modern climate patterns on the planet, began. Pars of India that are arid today seem to have had permanent water sources and much more rainfall before the Holocene Era.
\end{itemize}

\subsection{Classifying the Indian Stone Age}
\begin{itemize}
    \item Based on the type of stone tools used, the general level of technological advanement, and the geological era, the Indian stone age consists of the Paleolithic age (2 mya to 0.1 mya), \Gls{mesolithic age} (100,000 ya to 40,000 ya) and \Gls{neolithic age} (40,000 ya to 10,000 ya).
    \item Levels of plant and animal domestication are more difficult to gauge in an archaeological site, and thus less useful in dating human settlements. Parts of modern day India still follow a part-forager lifestyle.
\end{itemize}

\subsection{The Paleolithic Age}
\begin{itemize}
    \item Handaxes, cleavers and chopping tools are the most ubiquitous stone implements found in Lower Paleolithic sites in India (dated 0.4 mya to 0.1 mya), all over the Deccan plateau and the Sind region of Pakistan
    \item Major Paleolithic sites in India include the banks of Sabarmati in Gujarat, the Godavari river in Nashik, the Krishna river in Gulbarga and the Narmada river in Madhya Pradesh.
    \item Sandstone, limestone, quartzite and chert tend to be the most prevalent minerals used in the making of paleolithic stone tools. These sites often have plentiful sources of forage, water and fauna nearby, indicating their eligibility as settlements for early humans.
    \item The first evidence of human habitation in the Ganga river valley have been found in Jalaun district, UP, dating back to 45,000 ya (middle Paleolithic). The site features bone and stone tools, along with a large variety of game remains.
    \item By the Upper Paleolithic age, (40,000 ya to 10,000 ya), the predominant stone tools are small flake tools featuring narrow parallel sharp edges termed blades.
    \item Prehistoric art and \Gls{petroglyphs} mark the start of recorded human art in the Upper Paleolithic era. Objects that had no immediate utilitarian value, such as ornaments and ritualistic figurines are the most common artefacts found.
    \item Now debunked myths about hunter-gatherer societies include the idea that they were living a life of constant struggle for resources with little time left foe leisure, that hunting was the predominant sourec of nutrition over gathering and that their social hierarchies were rigid, with a defined chief and leaders.
\end{itemize}

\subsection{The Mesolithic Age}
\begin{itemize}
    \item The Mesolithic era reflects the start of sedentary human settlements, domestication of animals, even smaller sizes of stone tools, and rudimentary pottery.
    \item Mesolithic sites in India have mass burials with grave goods consisting of \glspl{microlith} and animal bone tools.
    \item Differences in the amount and nature of grave goods found at the different graves within Sarai Nahar Rai site in the Ganga valley suggest the existence of primitive hierarchies and the emergence of lineal claims on natural resources.
    \item The use of raw material not local to the tool-making site suggests that Mesolithic people travelled considerable distances to procure materials, either directly from the Earth, or through interaction with other tribal groups.
    \item Bhimbetka, in MP features one of the most elaborate rock art sites in the world, with beautiful Mesolithic paintings depicting daily life, animal motifs, and geometric shapes.
    \item Paintings could have been made to express creativity, to symbolize fertility and hunting rituals, to memorialize events in the life of the community, or simply as an expression of abstract ideas.
\end{itemize}

\section{The Transition to Food Production: Neolithic, Neolithic-Chalcolithic, and Chalcolithic Villages, c. 7000-2000 BCE}

\subsection{The Neolithic Age and the Beginnings of Food Production}
\begin{itemize}
    \item Plant and animal domestication requires the keeping aside of flora and fauna from food stores for eventual breeding or planting removed from their natural wild habitat.
    \item The earliest \glspl{food-producing society} are dated at 8000 BCE - 6000 BCE through sites found in Jordan, Iran, Turkey, Syria and Baluchistan province of Pakistan.
    \item This neolithic revilution involved the emergence of small, self-sufficient sedentary communites with division of labour based on sex and age.
\end{itemize}

\subsection{Why Domestication?}
\begin{itemize}
    \item The earliest argument for the origins of domestication was the climate at the end of the Pleistocene becoming more hostile leading to people concentrating in oases of favourable water, food and land resources.
    \item Another hypothesis looks at the people-food equilibrium at inland zones being upset by demographic pressure form coastal communities immigrating triggered by a rise in sea levels.
    \item The climate in the Holocene epoch becoming more hospitable for the wild growth and human cultivation of early food-grains like maize and barley is also a possible reason.
\end{itemize}

\subsection{The Identification of Domestication and Food Production in the Archaeological Record}
\begin{itemize}
    \item Evidence of domestication can be seen in the changes in animal \gls{morphology}, such as a weakening of muscles or shortening of bones, or in the sex ratios of animals being bred.
    \item Since evidence left behind by plants and animals domesticated in neolithic times is rare and unreliable, secondary evidence such as cave paintings, the usage of stone tools such as sickles grinding stones are used to corroborate finds of neolithic settlements.
    \item \Gls{paleobotany} uses charred remains of seeds and grains, or the hard outer shell of pollen (which is extraordinarily durable) to analyze archaeological sites. The newest methods include mass spectrometry and DNA analysis of fragments of the plant genome.
\end{itemize}

\subsection{The Transition to Food Production in the Indian Subcontinent}
\begin{itemize}
    \item 
\end{itemize}

